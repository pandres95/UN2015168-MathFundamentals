\input{../Common/preamble.tex.tpl}

\begin{document}

\maketitle

\begin{abstract}
Sesión dos - se trabajaran las nociones de validez y deducibilidad, enfatizando la relación entre estas dos nociones (T. Validez, T. Completitud)
\end{abstract}


\begin{enumerate}

\item Discuta con sus compañeros sobre la noción de validez y deducibilidad. Responda lo siguiente:
\begin{itemize}
    \item ¿Existe una relación entre validez y deducibilidad?
    \item Suponga un conjunto de premisas $\Lambda$, ¿ es posible deducir cualquier cosa a partir de este conjunto ?
    \item ¿Cuándo un conjunto de premisas $\Lambda$ se dice consistente ?
\end{itemize}

\item Valide si los siguientes conjuntos de premisas son consistentes.

\begin{multicols}{2}
\begin{itemize}
    \item $\{\neg p_{1} \wedge p_{2} \rightarrow p_{0}, p_{1} \rightarrow (\neg  p_{1} \rightarrow p_{2}), p_{0}  \leftrightarrow \neg p_{2}\}$
\end{itemize}
\columnbreak
\begin{itemize}
    \item $\{p_{0} \rightarrow p_{1}, p_{0} \wedge p_{2} \rightarrow p_{1} \wedge p_{3}, p_{0} \wedge p_{2} \wedge p_{4} \rightarrow p_{1} \wedge p_{3} \wedge p_{5}, ... \}$
\end{itemize}
\end{multicols}

\item Considere los siguientes enunciados, indique si los argumentos son validos y en los casos validos brinde una deducción. En este punto no se puede utilizar el método de reducción al absurdo.

\begin{itemize}
    \item Si alguien ve o alguien oye, hay alguien que está alerta. Si es cierto que alguien ve solo si no es posible la total sorpresa, entonces hay algo previsible. Luego, o bien nada hay previsible, o es posible la total sorpresa aunque haya alguien alerta.

    \item $ p_{1},p_{1} \rightarrow p, q_{1} \rightarrow q, p_{2} \rightarrow q_{1}, r \rightarrow p_{2} , p \rightarrow (q \vee  r ) \wedge \neg (q \wedge r )  \implies q \wedge  \neg r $

    \item $ \neg p \rightarrow (q \rightarrow \neg r), r \rightarrow \neg p, (\neg w \vee p) \rightarrow \neg \neg r, \neg w \implies \neg q $
    \item $ t \wedge w , q \rightarrow \neg t , \neg ( \neg p \wedge  \neg r ) , \neg q \rightarrow \neg p , r \rightarrow s \implies r $
\end{itemize}

\item Discuta con sus compañeros el método de reducción al absurdo, ¿por qué creen que funciona el método? Brinde deducciones a los problemas del punto anterior utilizando absurdo.

\end{enumerate}

\input{../Common/footer.tex.tpl}
