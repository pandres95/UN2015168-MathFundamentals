\input{../Common/preamble.tex.tpl}

\begin{document}

\maketitle

\begin{abstract}
Sesión cuatro - Preparcial.
\end{abstract}

\begin{enumerate}

    \item Brinde una deducción del siguiente argumento. Suponga que las funciones no tienen problemas de ningún tipo.
     $$\forall y \exists x (f(x) = y) , \forall y \exists x (g(x) = y) \implies \forall y \exists x (f(g(x)) = y ) $$

    \begin{solucion}

    Esta prueba tiene su detalle, hay que especificar el universal y luego el existencial, indicando que se tiene una relación entre los elementos.

    \begin{tabular}{c c c}
         (1) & $\forall y \exists x (f(x) = y)$ & Premisa \\
         (2) & $\forall y \exists x (g(x) = y)$ & Premisa \\

         (3) & $ \exists x (f(x) = b)$ & Para todo $b$ en el universo, especificación del universal en (1) \\
         (4) & $ (f(a) = b)$ & Para todo $b$ en el universo y algún $a$ que depende de $b$, especificación del existencial en (3) \\

         (5) & $ \exists x (g(x) = d)$ & Para todo $d$ en el universo, especificación del universal en (2) \\
         (6) & $ (g(c) = d)$ & Para todo $d$ en el universo y algún $c$ que depende de $d$, especificación del existencial en (5) \\

         (7) & $g(c) = a $ & Para todo $a$ en el universo y algún $c$ que depende de $a$, caso particular cuando $d= a$ en (6). \\
         (8) & $f(g(c)) = b$ & Para todo $b$ en el universo y algún $c$ que depende de $b$, remplazo de $a$ según (7) en (4) \\

         (9) & $\exists x (f(g(x)) = b)$ & Para todo $b$ en el universo, generalización del existencial en (8) \\
         (10) & $\forall y \exists x  (f(g(x)) = b)$ & Generalización del universal en (9)



    \end{tabular}

    \end{solucion}



    \item Escriba en correcto español, considere el universo de los naturales.
    \begin{multicols}{2}
    \begin{itemize}
        \item $p \vee t \rightarrow ( q \rightarrow s)  $
    \end{itemize}

    \columnbreak

    \begin{itemize}
        \item $\forall k ( p(k) \rightarrow p(k+1)) \rightarrow \forall n ( p(n)) $
    \end{itemize}
    \end{multicols}


    \begin{solucion}

    (1) Si, p o q entonces, s es necesario para p.


    (2) Tomando como referencia el universo de los naturales y pensando en una propiedad cualquiera se puede indicar de la siguiente manera:

    Si, para todo natural cada vez que se cumple la propiedad para $n$ implica la propiedad para $n+1$. Entonces todo natural cumple la propiedad.

    \end{solucion}

    \item Valide si los siguientes argumentos son validos, consistentes y brinde una deducción para aquellos validos.

    \begin{multicols}{2}
        \begin{center}
        \begin{tabular}{c}
             $\neg (p \vee q ) $\\
             $r \vee t $ \\
             $\neg q \rightarrow \neg r$ \\
             $t \rightarrow p$ \\
             \hline
             $p \leftrightarrow t \wedge r $
        \end{tabular}
        \end{center}
    \columnbreak
        \begin{center}
        \begin{tabular}{c}
            $(r \vee s ) \rightarrow ( t \wedge u ))$ \\
            $ \neg r \rightarrow (v \rightarrow \neg v) $ \\
            $\neg t$ \\
            \hline
            $\neg v $

        \end{tabular}
        \end{center}
    \end{multicols}

    \begin{solucion}

    \begin{itemize}
        \item Las premisas son inconsistentes y por lo tanto el argumento es valido. Existe una derivación, pero esto es inmediato de la inconsistencia.

        \item Las premisas son consistentes, tome $v,r,s$ falsos y $t$ verdadero. El argumento es válido, y por lo tanto existe una derivación de la conclusión.
    \end{itemize}
    \end{solucion}


    \item Determine si es tautología.

    \begin{multicols}{2}
    \begin{itemize}
        \item $(l \rightarrow (m\rightarrow n)) \rightarrow ( m \rightarrow (l \rightarrow n ))$
    \end{itemize}
    \columnbreak
    \end{multicols}

    \begin{solucion}
    Para que la proposición sea una tautología se necesita que el antecedente sea verdadero y el consecuente falso. Analice el consecuente, éste debe ser falso por lo que la única opción posible es que $m$ y $l$ son verdaderos y $n$ falso, pero estos valores hacen el antecedente falso. Por lo tanto no existen valores que hagan la proposición falsa.

    \end{solucion}

    \item Sean $n,m$ enteros, si $nm$ es impar, entonces $n$ y $m$ son impares.

    \begin{solucion}
        Se procede por reducción al absurdo. Suponga que $nm$ es impar y \emph{ $n$ o $m$ es par}
        \begin{itemize}
            \item Si $n$ es par y $m$ es impar, entonces $n = 2k $ para algún $K$ entero y $m = 2l+1$ para algún $l$ entero. Con lo cual $nm = (2k)(2l+1) = 2(k)(2l+1)$ usando asociatividad en el producto de enteros, observe que $k(2l+1)$ es entero y por lo tanto $nm = 2h$ con $h$ entero. De lo cual $nm$ es par y esto contradice el hecho que sea impar. No es posible que se de este caso.

            \item Si $n$ es impar y $m$ es par, la prueba es análoga.
            \item Si ambos son pares, prueba análoga.
        \end{itemize}

        Al analizar los tres se llega a un absurdo, por lo tanto la afirmación original debe ser verdad.
    \end{solucion}
\end{enumerate}

\input{../Common/footer.tex.tpl}
